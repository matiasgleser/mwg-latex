% Nothing in Introduction
\addtocounter{section}{1}

\section{Aggregate Demand and Aggregate Wealth}

\begin{prop}
    A necessary and sufficient condition for the set of consumers to exhibit parallel, straight wealth expansion paths at any price vector $p$ is that preferences admit indirect utility functions of the Gorman form with the coefficients on $w_i$ the same for every consumer $i$. That is:
    \begin{equation*}
        v_i(p, w_i) = a_i (p) + b(p)w_i.
    \end{equation*}
\end{prop}


\section{Aggregate Demand and the Weak Axiom}

\begin{defn}
    The aggregate demand function $x(p, w)$ satisfies the weak axiom (WA) if $p \cdot x(p', w') \leq w$ and $x(p, w) \neq x(p', w')$ imply $p' \cdot x(p, w) > w'$ for any $(p, w)$ and $(p', w')$.
\end{defn}

\begin{defn}
    The individual demand function $x_i(p, w_i)$ satisfies the \emph{uncompensated law of demand (ULD)} property if
    \begin{equation*}
        (p' - p) \cdot \left[x_i(p', w_i) - x_i(p, w_i) \right] \leq 0
    \end{equation*}
    for any $p, p'$, and $w_i$, with strict inequality if $x_i(p', w_i) \neq x_i(p, w_i)$. The analogous definition applies to the aggregate demand function $x(p, w)$.
\end{defn}

\begin{prop}
    If every consumer's Walrasian demand function $x_i(p, w_i)$ satisfies the uncompensated law of demand (ULD) property, so does the aggregate demand $x(p, w) = \sum_i x_i(p, \alpha_i w)$. As a consequence, the aggregate demand $x(p, w)$ satisfies the weak axiom.
\end{prop}

\begin{prop}
    If $\succsim_i$ is homothetic, then $x_i(p, w_i)$ satisfies the uncompensated law of demand (ULD) property.
\end{prop}

\begin{prop}
    Suppose that $\succsim_i$ is defined on the consumption set $X = \mathbb{R}^L_+$ and is representable by a twice continuously differentiable concave function $u_i(\cdot)$. If 
    \begin{equation*}
        - \frac{x_i \cdot D^2 u_i(x_i) x_i}{x_i \cdot \nabla u_i(x_i)} < 4 \quad \text{for all } x_i,
    \end{equation*}
    then $x_i(p, w_i)$ satisfies the unrestricted law of demand (ULD) property.
\end{prop}

\begin{prop}
    Suppose that all consumers have identical preferences $\succsim$ defined on $\mathbb{R}^{L}_+$ [with individual demand functions denoted by $\tilde{x}(p, w)$] and that individual wealth is uniformly distributed on an interval $[0, \bar{w}]$ (strictly speaking this requires a continuum of consumers). Then the aggregate (rigorously, the average) demand function
    \begin{equation*}
        x(p) = \int_0^{\bar{w}} \tilde{x}(p, w) dw
    \end{equation*}
    satisfies the unrestricted law of demand (ULD) property.
\end{prop}


\section{Aggregate Demand and the Existence of a Representative Consumer}

\begin{defn}
    A \emph{positive representative consumer} exists of there is a rational preference relation $\succsim$ on $\mathbb{R}^{L}_+$ such that the aggregate demand function $x(p, w)$ is precisely the Walrasian demand function generated by this preference relation. That is, $x(p, w) \succ x$ whenever $x \neq x(p, w)$ and $p \cdot x \leq w$.
\end{defn}

\begin{defn}
    A \emph{(Berson-Samuelson) social welfare function} is a function $W: \mathbb{R}^{I} \to \mathbb{R}$ that assigns a utility value to each possible vector $(u_1, \dots, u_I) \in \mathbb{R}^{I}$ of utility levels for the $I$ consumers in the economy.
\end{defn}

\begin{prop}
    Suppose that for each level of prices $p$ and aggregate wealth $w$, the wealth distribution $w_1(p, w), \dots, w_I(p, w)$ solves 
    \begin{equation}\label{pi.chiv.welfare-max}
        \begin{aligned}
            \max_{w_1, \dots, w_I} &W\left(v_1(p, w_1), \dots, v_I(p, w_I) \right) \\
            & \text{s.t. } \sum_{i = 1}^I w_i \leq w.
        \end{aligned}
    \end{equation}
    Then the value function $v(p, w)$ of problem (\ref{pi.chiv.welfare-max}) is an indirect utility function of a positive representative consumer for the aggregate demand function $x(p, w) = \sum_i x_i(p, w_i(p, w))$.
\end{prop}

\begin{defn}
    The positive representative consumer $\succsim$ for the aggregate demand $x(p, w) = \sum_i x_i(p, w_i(p, w))$ is a \emph{normative representative consumer} relative to the social welfare function $W(\cdot)$ if for every $(p, w)$, the distribution of wealth $w_1(p, w), \dots, w_I(p, w)$ solves problem (\ref{pi.chiv.welfare-max}) and, therefore, the value function of problem (\ref{pi.chiv.welfare-max}) is an indirect utility function for $\succsim$.
\end{defn}
