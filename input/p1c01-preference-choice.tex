% Nothing in Introduction
\addtocounter{section}{1}

\section{Preference Relations}

\begin{defn}
    The preference relation $\succsim$ is \emph{rational} if it possesses the following two properties:
    \begin{enumerate}
        \item \emph{Completeness}: for all $x, y \in X$ we have that $x \succsim y$ or $y \succsim x$ (or both).
        \item \emph{Transitivity}: For all $x, y, z \in X$, if $x \succsim y$ and $y \succsim z$, then $x \succsim z$.
    \end{enumerate}
\end{defn}

\begin{prop}
    If $\succsim$ is rational, then 
    \begin{enumerate}
        \item $\succ$ is both \emph{irreflexive} ($x \succ x$ never holds) and \emph{transitive} (if $x \succ y$ and $y \succ z$, then $x \succ z$).
        \item $\sim$ is \emph{reflexive} ($x \sim x$ for all $x$), \emph{transitive} (if $x \sim y$ and $y \sim z$, then $x \sim z$), and \emph{symmetric} (if $x \sim y$, then $y \sim x$).
        \item If $x \succ y \succsim z$ then $x \succ z$.
    \end{enumerate}
\end{prop}

\begin{defn}
    A function $u: X \to \reals$ is a \emph{utility function representing $\succsim$} if, for all $x, y \in X$,
    \begin{equation*}
        x \succsim y \iff u(x) \geq u(y).
    \end{equation*}
\end{defn}

\begin{prop}
    A preference relation $\succsim$ can be represented by a utility function only if it is rational.
\end{prop}

\section{Choice Rules}

\begin{defn}
    The choice structure $(\mathscr{B}, C(\cdot))$ satisfies the \emph{weak axiom of revealed preference} if the following property holds:

    If for some $B \in \mathscr{B}$ with $x, y \in B$ we have $x \in C(B)$, then for any $B' \in \mathscr{B}$ with $x, y \in B'$ and $y \in C(B')$, we must also have $x \in C(B')$.
\end{defn}

\begin{defn}
    Given a choice structure $(\mathscr{B}, C(\cdot))$ the \emph{revealed preference relation} $\succsim^*$ is defined by 

    $x \succsim^* y \iff$ there is some $B \in \mathscr{B}$ such that $x, y \in B$ and $x \in C(B)$.
\end{defn}

\section{The Relationship between Preference Relations and Choice Rules}

\begin{prop}
    Suppose that $\succsim$ is a rational preference relation. Then the choice structure generated by $\succsim$, $(\mathscr{B}, C^*(\cdot, \succsim))$ satisfies the weak axiom.
\end{prop}

\begin{defn}
    Given a choice structure $(\mathscr{B}, C(\cdot))$, we say that the rational preference relation $\succsim$ \emph{rationalises} $C(\cdot)$ relative to $\mathscr{B}$ if
    \begin{equation*}
        C(B) = C^*(B, \succsim)
    \end{equation*}
    for all $B \in \mathscr{B}$, that is, if $\succsim$ generates the choice structure $(\mathscr{B}, C(\cdot))$.
\end{defn}

\begin{prop}
    If $(\mathscr{B}, C(\cdot))$ is a choice structure such that
    \begin{enumerate}
        \item the weak axiom is satisfied,
        \item $\mathscr{B}$ includes all subsets of $X$ of up to three elements,
    \end{enumerate}
    then there is a rational preference relation $\succsim$ that rationalises $C(\cdot)$ relative to $\mathscr{B}$; that is, $C(B) = C^*(B, \succsim)$, for all $B \in \mathscr{B}$. Furthermore, this rational preference relation is the \emph{only} preference relation that does.
\end{prop}
